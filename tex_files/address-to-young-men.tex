\documentclass[a4paper]{article}
\usepackage{setspace}
\usepackage[fontsize=14pt]{fontsize}
\usepackage[cmintegrals,cmbraces]{newtxmath}
\usepackage{ebgaramond-maths}
\usepackage[T1]{fontenc}
\usepackage{geometry}
\usepackage{fancyhdr}
\usepackage{enumerate}
\usepackage{lettrine}
\usepackage[ddmmyyyy]{datetime}

\geometry{textwidth=460pt, textheight=700pt}

\pagestyle{fancy}
\fancyhf{}
\rhead{\textit{Address to Young Men on the Right Use of Greek Literature}}
\lhead{Saint Basil the Great}
\rfoot{\thepage}

\makeatletter
\renewcommand{\maketitle}{\bgroup\setlength{\parindent}{0pt}
\begin{flushleft}
  \huge{\textbf{\@title}}
  
  \vspace{0.2cm}
  
  \large{\@author}
  
  \vspace{0.1cm}
  
  \normalsize{\@date}
\end{flushleft}\egroup
}
\makeatother

\title{Politics and the English Language}
\author{George Orwell}
\date{1946}

\begin{document}

\begin{titlepage}

 \hrule
 
  \vspace*{120pt}% Similar to a regular \chapter gap
  
  \centering

  \HUGE \textbf{\textit{Address to Young Men on the Right Use of Greek Literature}}

  \bigskip

  \LARGE Saint Basil the Great \smallskip \\ 
  \normalsize (Basil of Caesarea) % Name

  \smallskip

  \large 330 - 379 A.D. % Date

  \vspace{250pt}% Vertical fill to go to the bottom of the page

  \normalsize
  The Open Wisdom Project
  \footnotesize
  
  Started on: 21/02/2022
  
  Completed on: \today
  
  Revised on: \today
  
  \vfill
  
  \hrule
\end{titlepage}

\normalsize

\begin{center}
 \textbf{Outline}
\end{center}


\begin{enumerate}[I.]
 
 \item Introduction: Out of the abundance of his experience the author will advise young men as to the pagan literature, showing them what to accept, and what to reject.
 
 \item To the Christian the life eternal is the supreme goal, and the guide to this life is the Holy Scriptures; but since young men cannot appreciate the deep thoughts contained therein, they are to study the profane writings, in which truth appears as in a mirror.
 
 \item Profane learning should ornament the mind, as foliage graces the fruit-bearing tree.
 
 \item In studying pagan lore one must discriminate between the helpful and the injurious, accepting the one, but closing one's ears to the siren song of the other.
 
 \item Since the life to come is to be attained through virtue, chief attention must be paid to those passages in which virtue is praised; such may be found, for example, in Hesiod, Homer, Solon, Theognis, and Prodicus.
 
 \item Indeed, almost all eminent philosophers have extolled virtue. The words of such men should meet with more than mere theoretical acceptance, for one must try to realize them in his life, remembering that to seem to be good when one is not so is the height of injustice.
 
 \item But in the pagan literature virtue is lauded in deeds as well as in words, wherefore one should study those acts of noble men which coincide with the teachings of the Scriptures.
 
 \item To return to the original thought, young men must distinguish between helpful and injurious knowledge, keeping clearly in mind the Christian's purpose in life. So, like the athlete or the musician, they must bend every energy to one task, the winning of the heavenly crown.
 
 \item This end is to be compassed by holding the body under, by scorning riches and fame, and by subordinating all else to virtue. 
 
 \item While this ideal will be matured later by the study of the Scriptures, it is at present to be fostered by the study of the pagan writers; from them should be stored up knowledge for the future.
 
 \item Conclusion: The above are some of the more important precepts; others the writer will continue to explain from time to time, trusting that no young man will make the fatal error of disregarding them. 
 
\end{enumerate}

\newpage

\lettrine[lines=3, findent=3pt, nindent=0pt]{I.} Many considerations, young men, prompt me to recommend to you the principles which I deem most desirable, and which I believe will be of use to you if you will adopt them. For my time of life, my many-sided training, yea, my adequate experience in those vicissitudes of life which teach their lessons at every turn, have so familiarized me with human affairs, that I am able to map out the safest course for those just starting upon their careers. By nature's common bond I stand in the same relationship to you as your parents, so that I am no whit behind them in my concern for you. Indeed, if I do not misinterpret your feelings, you no longer crave your parents when you come to me. Now if you should receive my words with gladness, you would be in the second class of those who, according to Hesiod, merit praise; if not, I should say nothing disparaging, but no doubt you yourselves would remember the passage in which that poet says: ``He is best who, of himself, recognizes what is his duty, and he also is good who follows the course marked out by others, but he who does neither of these things is of no use under the sun.''

Do not be surprised if to you, who go to school every day, and who, through their writings, associate with the learned men of old, I say that out of my own experience I have evolved something more useful. Now this is my counsel, that you should not unqualifiedly give over your minds to these men, as a ship is surrendered to the rudder, to follow whither they list, but that, while receiving whatever of value they have to offer, you yet recognize what it is wise to ignore. Accordingly, from this point on I shall take up and discuss the pagan writings, and how we are to discriminate among them.

\bigskip

\lettrine[lines=3, findent=3pt, nindent=0pt]{II.} We Christians, young men, hold that this human life is not a supremely precious thing, nor do we recognize anything as unconditionally a blessing which benefits us in this life only. Neither pride of ancestry, nor bodily strength, nor beauty, nor greatness, nor the esteem of all men, nor kingly authority, nor, indeed, whatever of human affairs may be called great, do we consider worthy of desire, or the possessors of them as objects of envy; but we place our hopes upon the things which are beyond, and in preparation for the life eternal do all things that we do. Accordingly, whatever helps us towards this we say that we must love and follow after with all our might, but those things which have no bearing upon it should be held as naught. But to explain what this life is, and in what way and manner we shall live it, requires more time than is at our command, and more mature hearers than you.

And yet, in saying thus much, perhaps I have made it sufficiently clear to you that if one should estimate and gather together all earthly weal from the creation of the world, he would not find it comparable to the smallest part of the possessions of heaven; rather, that all the precious things in this life fall further short of the least good in the other than the shadow or the dream fails of the reality. Or rather, to avail myself of a still more natural comparison, by as much as the soul is superior to the body in all things, by so much is one of these lives superior to the other.

Into the life eternal the Holy Scriptures lead us, which teach us through divine words. But so long as our immaturity forbids our understanding their deep thought, we exercise our spiritual perceptions upon profane writings, which are not altogether different, and in which we perceive the truth as it were in shadows and in mirrors. Thus we imitate those who perform the exercises of military practice, for they acquire skill in gymnastics and in dancing, and then in battle reap the reward of their training. We must needs believe that the greatest of all battles lies before us, in preparation for which we must do and suffer all things to gain power. Consequently, we must be conversant with poets, with historians, with orators, indeed with all men who may further our soul's salvation. Just as dyers prepare the cloth before they apply the dye, be it purple or any other color, so indeed must we also, if we would preserve indelible the idea of the true virtue, become first initiated in the pagan lore, then at length give special heed to the sacred and divine teachings, even as we first accustom ourselves to the sun's reflection in the water, and then become able to turn our eyes upon the very sun itself.

\bigskip

\lettrine[lines=3, findent=3pt, nindent=0pt]{III.} If, then, there is any affinity between the two literatures, a knowledge of them should be useful to us in our search for truth; if not, the comparison, by emphasizing the contrast, will be of no small service in strengthening our regard for the better one. With what now may we compare these two kinds of education to obtain a simile? Just as it is the chief mission of the tree to bear its fruit in its season, though at the same time it puts forth for ornament the leaves which quiver on its boughs, even so the real fruit of the soul is truth, yet it is not without advantage for it to embrace the pagan wisdom, as also leaves offer shelter to the fruit, and an appearance not untimely. That Moses, whose name is a synonym for wisdom, severely trained his mind in the learning of the Egyptians, and thus became able to appreciate their deity. Similarly, in later days, the wise Daniel is said to have studied the lore of the Chaldeans while in Babylon, and after that to have taken up the sacred teachings.

\bigskip

\lettrine[lines=3, findent=3pt, nindent=0pt]{IV.} Perhaps it is sufficiently demonstrated that such heathen learning is not unprofitable for the soul; I shall then discuss next the extent to which one may pursue it. To begin with the poets, since their writings are of all degrees of excellence, you should not study all of their poems without omitting a single word. When they recount the words and deeds of good men, you should both love and imitate them, earnestly emulating such conduct. But when they portray base conduct, you must flee from them and stop up your ears, as Odysseus is said to have fled past the song of the sirens, for familiarity with evil writings paves the way for evil deeds. Therefore the soul must be guarded with great care, lest through our love for letters it receive some contamination unawares, as men drink in poison with honey. We shall not praise the poets when they scoff and rail, when they represent fornicators and winebibbers, when they define blissfulness by groaning tables and wanton songs. Least of all shall we listen to them when they tell us of their gods, and especially when they represent them as being many, and not at one among themselves. For, among these gods, at one time brother is at variance with brother, or the father with his children; at another, the children engage in truceless war against their parents. The adulteries of the gods and their amours, and especially those of the one whom they call Zeus, chief of all and most high, things of which one cannot speak, even in connection with brutes, without blushing, we shall leave to the stage. I have the same words for the historians, and especially when they make up stories for the amusement of their hearers. And certainly we shall not follow the example of the rhetoricians in the art of lying. For neither in the courts of justice nor in other business affairs will falsehood be of any help to us Christians, who, having chosen the straight and true path of life, are forbidden by the gospel to go to law. But on the other hand we shall receive gladly those passages in which they praise virtue or condemn vice. For just as bees know how to extract honey from flowers, which to men are agreeable only for their fragrance and color, even so here also those who look for something more than pleasure and enjoyment in such writers may derive profit for their souls. Now, then, altogether after the manner of bees must we use these writings, for the bees do not visit all the flowers without discrimination, nor indeed do they seek to carry away entire those upon which they light, but rather, having taken so much as is adapted to their needs, they let the rest go. So we, if wise, shall take from heathen books whatever befits us and is allied to the truth, and shall pass over the rest. And just as in culling roses we avoid the thorns, from such writings as these we will gather everything useful, and guard against the noxious. So, from the very beginning, we must examine each of their teachings, to harmonize it with our ultimate purpose, according to the Doric proverb, ``testing each stone by the measuring-line.''

\bigskip

\lettrine[lines=3, findent=3pt, nindent=0pt]{V.} Since we must needs attain to the life to come through virtue, our attention is to be chiefly fastened upon those many passages from the poets, from the historians, and especially from the philosophers, in which virtue itself is praised. For it is of no small advantage that virtue become a habit with a youth, for the lessons of youth make a deep impression, because the soul is then plastic, and therefore they are likely to be indelible. If not to incite youth to virtue, pray what meaning may we suppose that Hesiod had in those universally admired lines, of which the sentiment is as follows: ``Rough is the start and hard, and the way steep, and full of labor and pain, that leads toward virtue. Wherefore, on account of the steepness, it is not granted to every man to set out, nor, to the one having set out, easily to reach the summit. But when he has reached the top, he sees that the way is smooth and fair, easy and light to the foot, and more pleasing than the other, which leads to wickedness,''—of which the same poet said that one may find it all around him in great abundance. Now it seems to me that he had no other purpose in saying these things than so to exhort us to virtue, and so to incite us to bravery, that we may not weaken our efforts before we reach the goal. And certainly if any other man praises virtue in a like strain, we will receive his words with pleasure, since our aim is a common one.

Now as I have heard from one skilful in interpreting the mind of a poet, all the poetry of Homer is a praise of virtue, and with him all that is not merely accessory tends to this end. There is a notable instance of this where Homer first made the princess reverence the leader of the Cephallenians, though he appeared naked, shipwrecked, and alone, and then made Odysseus as completely lack embarrassment, though seen naked and alone, since virtue served him as a garment. And next he made Odysseus so much esteemed by the other Phaeacians that, abandoning the luxury in which they lived, all admired and emulated him, and there was not one of them who longed for anything else except to be Odysseus, even to the enduring of shipwreck. The interpreter of the poetic mind argued that, in this episode, Homer very plainly says: ``Be virtue your concern, men, which both swims to shore with the shipwrecked man, and makes him, when he comes naked to the strand, more honored than the prosperous Phaeacians.'' And, indeed, this is the truth, for other possessions belong to the owner no more than to another, and, as when men are dicing, fall now to this one, now to that. But virtue is the only possession that is sure, and that remains with us whether living or dead. Wherefore it seems to me that Solon had the rich in mind when he said: ``We will not exchange our virtue for their gold, for virtue is an everlasting possession, while riches are ever changing owners.'' Similarly Theognis said that the god, whatever he might mean by the god, inclines the balances for men, now this way, now that, giving to some riches, and to others poverty. Also Prodicus, the sophist of Ceos, whose opinion we must respect, for he is a man not to be slighted, somewhere in his writings expressed similar ideas about virtue and vice. I do not remember the exact words, but as far as I recollect the sentiment, in plain prose it ran somewhat as follows: While Hercules was yet a youth, being about your age, as he was debating which path he should choose, the one leading through toil to virtue, or its easier alternate, two women appeared before him, who proved to be Virtue and Vice. Though they said not a word, the difference between them was at once apparent from their mien. The one had arranged herself to please the eye, while she exhaled charms, and a multitude of delights swarmed in her train. With such a display, and promising still more, she sought to allure Hercules to her side. The other, wasted and squalid, looked fixedly at him, and bespoke quite another thing. For she promised nothing easy or engaging, but rather infinite toils and hardships, and perils in every land and on every sea. As a reward for these trials, he was to become a god, so our author has it. The latter, Hercules at length followed.

\bigskip

\lettrine[lines=3, findent=3pt, nindent=0pt]{VI.} Almost all who have written upon the subject of wisdom have more or less, in proportion to their several abilities, extolled virtue in their writings. Such men must one obey, and must try to realize their words in his life. For he, who by his works exemplifies the wisdom which with others is a matter of theory alone, ``breathes; all others flutter about like shadows.'' I think it is as if a painter should represent some marvel of manly beauty, and the subject should actually be such a man as the artist pictures on the canvas. To praise virtue in public with brilliant words and with long drawn out speeches, while in private preferring pleasures to temperance, and self-interest to justice, finds an analogy on the stage, for the players frequently appear as kings and rulers, though they are neither, nor perhaps even genuinely free men. A musician would hardly put up with a lyre which was out of tune, nor a choregus with a chorus not singing in perfect harmony. But every man is divided against himself who does not make his life conform to his words, but who says with Euripides, ``The mouth indeed hath sworn, but the heart knows no oath.'' Such a man will seek the appearance of virtue rather than the reality. But to seem to be good when one is not so, is, if we are to respect the opinion of Plato at all, the very height of injustice.

\bigskip

\lettrine[lines=3, findent=3pt, nindent=0pt]{VII.} After this wise, then, are we to receive those words from the pagan authors which contain suggestions of the virtues. But since also the renowned deeds of the men of old either are preserved for us by tradition, or are cherished in the pages of poet or historian, we must not fail to profit by them. A fellow of the street rabble once kept taunting Pericles, but he, meanwhile, gave no heed; and they held out all day, the fellow deluging him with reproaches, but he, for his part, not caring. Then when it was evening and dusk, and the fellow still clung to him, Pericles escorted him with a light, in order that he might not fail in the practice of philosophy. Again, a man in a passion threatened and vowed death to Euclid of Megara, but he in turn vowed that the man should surely be appeased, and cease from his hostility to him.

How invaluable it is to have such examples in mind when a man is seized with anger! On the other hand, one must altogether ignore the tragedy which says in so many words: ``Anger arms the hand against the enemy;'' for it is much better not to give way to anger at all. But if such restraint is not easy, we shall at least curb our anger by reflection, so as not to give it too much rein.

But let us bring our discussion back again to the examples of noble deeds. A certain man once kept striking Socrates, the son of Sophroniscus, in the face, yet he did not resent it, but allowed full play to the ruffian's anger, so that his face was swollen and bruised from the blows. Then when he stopped striking him, Socrates did nothing more than write on his forehead, as an artisan on a statue, who did it, and thus took out his revenge. Since these examples almost coincide with our teachings, I hold that such men are worthy of emulation. For this conduct of Socrates is akin to the precept that to him who smites you upon the one cheek, you shall turn the other also — thus much may you be avenged; the conduct of Pericles and of Euclid also conforms to the precept: ``Submit to those who persecute you, and endure their wrath with meekness;'' and to the other: ``Pray for your enemies and curse them not.'' One who has been instructed in the pagan examples will no longer hold the Christian precepts impracticable. But I will not overlook the conduct of Alexander, who, on taking captive the daughters of Darius, who were reputed to be of surpassing beauty, would not even look at them, for he deemed it unworthy of one who was a conqueror of men to be a slave to women. This is of a piece with the statement that he who looks upon a woman to lust after her, even though he does not commit the act of adultery, is not free from its guilt, since he has entertained impure thoughts. It is hard to believe that the action of Cleinias, one of the disciples of Pythagoras, was in accidental conformity to our teachings, and not designed imitation of them. What, then, was this act of his? By taking an oath he could have avoided a fine of three talents, yet rather than do so he paid the fine, though he could have sworn truthfully. I am inclined to think that he had heard of the precept which forbids us to swear.

\bigskip

\lettrine[lines=3, findent=3pt, nindent=0pt]{VIII.} But let us return to the same thought with which we started, namely, that we should not accept everything without discrimination, but only what is useful. For it would be shameful should we reject injurious foods, yet should take no thought about the studies which nourish our souls, but as a torrent should sweep along all that came near our path and appropriate it. If the helmsman does not blindly abandon his ship to the winds, but guides it toward the anchorage; if the archer shoots at his mark; if also the metal-worker or the carpenter seeks to produce the objects for which his craft exists, would there be rime or reason in our being outclassed by these men, mere artisans as they are, in quick appreciation of our interests? For is there not some end in the artisan's work, is there not a goal in human life, which the one who would not wholly resemble unreasoning animals must keep before him in all his words and deeds? If there were no intelligence sitting at the tiller of our souls, like boats without ballast we should be borne hither and thither through life, without plan or purpose.

An analogy may be found in the athletic contests, or, if you will, in the musical contests; for the contestants prepare themselves by a preliminary training for those events in which wreaths of victory are offered, and no one by training for wrestling or for the pancratium would get ready to play the lyre or the flute. At least Polydamas would not, for before the Olympic games he was wont to bring the rushing chariot to a halt, and thus hardened himself. Then Milo could not be thrust from his smeared shield, but, shoved as he was, clung to it as firmly as statues soldered by lead. In a word, by their training they prepared themselves for the contests. If they had meddled with the airs of Marsyas or of Olympus, the Phrygians, abandoning dust and exercise, would they have won ready laurels or crowns, or would they have escaped being laughed at for their bodily incapacity? On the other hand, certainly Timotheus the musician did not spend his time in the schools for wrestling, for then it would not have been his to excel all in music, he who was so skilled in his art that at his pleasure he could arouse the passions of men by his harsh and vehement strains, and then by gentle ones, quiet and soothe them. By this art, when once he played Phrygian airs on the flute to Alexander, he is said to have incited the general to arms in the midst of feasting, and then, by milder music, to have restored him to his carousing friends. Such power to compass one's end, either in music or in athletic contests, is developed by practice.

I have called to mind the wreaths and the fighters. These men endure hardships beyond number, they use every means to increase their strength, they sweat ceaselessly at their training, they accept many blows from the master, they adopt the mode of life which he prescribes, though it is most unpleasant, and, in a word, they so rule all their conduct that their whole life before the contest is preparatory to it. Then they strip themselves for the arena, and endure all and risk all, to receive the crown of olive, or of parsley, or some other branch, and to be announced by the herald as victor.

Will it then be possible for us, to whom are held out rewards so wondrous in number and in splendor that tongue can not recount them, while we are fast asleep and leading care-free lives, to make these our own by half-hearted efforts? Surely, were an idle life a very commendable thing, Sardanapalus would take the first prize, or Margites if you will, whom Homer, if indeed the poem is by Homer, put down as neither a farmer, nor a vine-dresser, nor anything else that is useful. Is there not rather truth in the maxim of Pittacus which says, ``It is hard to be good?'' For after we have actually endured many hardships, we shall scarcely gain those blessings to which, as said above, nothing in human experience is comparable. Therefore we must not be light-minded, nor exchange our immortal hopes for momentary idleness, lest reproaches come upon us, and judgment befall us, not forsooth here among men, although judgment here is no easy thing for the man of sense to bear, but at the bar of justice, be that under the earth, or wherever else it may happen to be. While he who unintentionally violates his obligations perchance receives some pardon from God, he who designedly chooses a life of wickedness doubtless has a far greater punishment to endure. 

\bigskip

\lettrine[lines=3, findent=3pt, nindent=0pt]{IX.} ``What then are we to do?'' perchance someone may ask. What else than to care for the soul, never leaving an idle moment for other things? Accordingly, we ought not to serve the body any more than is absolutely necessary, but we ought to do our best for the soul, releasing it from the bondage of fellowship with the bodily appetites; at the same time we ought to make the body superior to passion. We must provide it with the necessary food, to be sure, but not with delicacies, as those do who seek everywhere for waiters and cooks, and scour both earth and sea, like those bringing tribute to some stern tyrant. This is a despicable business, in which are endured things as unbearable as the torments of hell, where wool is combed into the fire, or water is drawn in a sieve and poured into a perforated jar, and where work is never done. Then to spend more time than is necessary on one's hair and clothes is, in the words of Diogenes, the part of the unfortunate or of the sinful. For what difference does it make to a sensible man whether he is clad in a robe of state or in an inexpensive garment, so long as he is protected from heat and cold? Likewise in other matters we must be governed by necessity, and only give so much care to the body as is beneficial to the soul. For to one who is really a man it is no less a disgrace to be a fop or a pamperer of the body than to be the victim of any other base passion. Indeed, to be very zealous in making the body appear very beautiful is not the mark of a man who knows himself, or who feels the force of the wise maxim : ``Not that which is seen is the man,'' for it requires a higher faculty for any one of us, whoever he may be, to know himself. Now it is harder for the man who is not pure in heart to gain this knowledge than for a blear-eyed person to look upon the sun. 

To speak generally and so far as your needs demand, purity of soul embraces these things: to scorn sensual pleasures, to refuse to feast the eyes on the senseless antics of buffoons, or on bodies which goad one to passion, and to close one's ears to songs which corrupt the mind. For passions which are the offspring of servility and baseness are produced by this kind of music. On the other hand, we must employ that class of music which is better in itself and which leads to better things, which David, the sacred psalmist, is said to have used to assuage the madness of the king. Also tradition has it that when Pythagoras happened upon some drunken revelers, he commanded the flute-player, who led the merry-making, to change the tune and to play a Doric air, and that the chant so sobered them that they threw down their wreaths, and shamefacedly returned home. Others at the sound of the flute rave like Corybantes and Bacchantes. Even so great a difference does it make whether one lends his ear to healthy or to vicious music. Therefore you ought to have still less to do with the music of such influence than with other infamous things. Then I am ashamed to forbid you to load the air with all kinds of sweet-smelling perfumes, or to smear yourselves with ointment. Again, what further argument is needed against seeking the gratification of one's appetite than that it compels those who pursue it, like animals, to make of their bellies a god? 

In a word, he who would not bury himself in the mire of sensuality must deem the whole body of little worth, or must, as Plato puts it, pay only so much heed to the body as is an aid to wisdom, or as Paul admonishes somewhere in a similar passage: ``Let no one make provision for the flesh, to fulfill the lusts thereof.'' Wherein is there any difference between those who take pains that the body shall be perfect, but ignore the soul, for the use of which it is designed, and those who are scrupulous about their tools, but neglectful of their trade? On the contrary, one ought to discipline the flesh and hold it under, as a fierce animal is controlled, and to quiet, by the lash of reason, the unrest which it engenders in the soul, and not, by giving full rein to pleasure, to disregard the mind, as a charioteer is run away with by unmanageable and frenzied horses. So let us bear in mind the remark of Pythagoras, who, upon learning that one of his followers was growing very fleshy from gymnastics and hearty eating, said to him, ``Will you not stop making your imprisonment harder for yourself?'' Then it is said that since Plato foresaw the dangerous influence of the body, he chose an unhealthy part of Athens for his Academy, in order to remove excessive bodily comfort, as one prunes the rank shoots of the vines. Indeed I have even heard physicians say that over-healthiness is dangerous. Since, then, this exaggerated care of the body is harmful to the body itself, and a hindrance to the soul, it is sheer madness to be a slave to the body, and serve it. 

If we were minded to disregard attention to the body, we should be in little danger of prizing anything else unduly. For of what use, now, are riches, if one scorns the pleasures of the flesh? I certainly see none, unless, as in the case of the mythological dragons, there is some satisfaction in guarding hidden treasure. Of a truth, one who had learned to be independent of this sort of thing would be loath to attempt anything mean or low, either in word or deed. For superfluity, be it Lydian gold-dust, or the work of the gold-gathering ants, he would disdain in proportion to its needlessness, and of course he would make the necessities of life, not its pleasures, the measure of need. Forsooth, those who exceed the bounds of necessity, like men who are sliding down an inclined plane, can nowhere gain a footing to check their precipitous flight, for the more they can scrape together, so much or even more do they need for the gratification of their desires. As Solon, the son of Execestides, puts it, ``No definite limit is set to a man's wealth.'' Also, one should hear Theognis, the teacher, on this point: ``I do not long to be rich, nor do I pray for riches, but let it be given me to live with a little, suffering no ill.'' 

I also admire the wholesale contempt of all human possessions which Diogenes expressed, who showed himself richer than the great Persian king, since he needed less for living. But we are wont to be satisfied with nothing save with the talents of the Mysian Pythius, with limitless acres of land, and more herds of cattle than may be counted. Yet I believe that if riches fail us we should not mourn for them, and if we have them, we should not think more of possessing them than of using them rightly. For Socrates expressed an admirable thought when he said that a rich, purse-proud man was never an object of admiration with him until he learned that the man knew how to use his wealth. If Phidias and Polycletus had been very proud of the gold and ivory with which the one constructed the statue of the Jupiter of Elis, the other the Juno of Argos, they would have been laughed at, because priding themselves in treasure produced by no merit of theirs, and overlooking their art, from which the gold gained greater beauty and worth. Then shall we think that we are open to less reproach if we hold that virtue is not, in and of itself, a sufficient ornament? Again, shall we, while manifestly ignoring riches and scorning sensual pleasures, court adulation and fulsome praise, vying with the fox of Archilochus in cunning and craft? Of a truth there is nothing which the wise man must more guard against than the temptation to live for praise, and to study what pleases the crowd. Rather truth should be made the guide of one's life, so that if one must needs speak against all men, and be in ill-favor and in danger for virtue's sake, he shall not swerve at all from that which he considers right; else how shall we say that he differs from the Egyptian sophist, who at pleasure turned himself into a tree, an animal, fire, water, or anything else? Such a man now praises justice to those who esteem it, and now expresses opposite sentiments when he sees that wrong is in good repute; this is the fawner's trick. Just as the polypus is said to take the color of the ground upon which it lies, so he conforms his opinions to those of his associates. 

\bigskip

\lettrine[lines=3, findent=3pt, nindent=0pt]{X.} To be sure, we shall become more intimately acquainted with these precepts in the sacred writings, but it is incumbent upon us, for the present, to trace, as it were, the silhouette of virtue in the pagan authors. For those who carefully gather the useful from each book are wont, like mighty rivers, to gain accessions on every hand. For the precept of the poet which bids us add little to little must be taken as applying not so much to the accumulation of riches, as of the various branches of learning. In line with this Bias said to his son, who, as he was about to set out for Egypt, was inquiring what course he could pursue to give his father the greatest satisfaction: ``Store up means for the journey of old age.'' By means he meant virtue, but he placed too great restrictions upon it, since he limited its usefulness to the earthly life. For if any one mentions the old age of Tithonus, or of Arganthonius, or of that Methuselah who is said to have lacked but thirty years of being a millenarian, or even if he reckons the entire period since the creation, I will laugh as at the fancies of a child, since I look forward to that long, undying age, of the extent of which there is no limit for the mind of man to grasp, any more than there is of the life immortal. For the journey of this life eternal I would advise you to husband resources, leaving no stone unturned, as the proverb has it, whence you might derive any aid. From this task we shall not shrink because it is hard and laborious, but, remembering the precept that every man ought to choose the better life, and expecting that association will render it pleasant, we shall busy ourselves with those things that are best. For it is shameful to squander the present, and later to call back the past in anguish, when no more time is given. 

\bigskip

In the above treatise I have explained to you some of the things which I deem the most to be desired; of others I shall continue to counsel you so long as life is allowed me. Now as the sick are of three classes, according to the degrees of their sickness, may you not seem to belong to the third, or incurable, class, nor show a spiritual malady like that of their bodies! For those who are slightly indisposed visit physicians in person, and those who are seized by violent sickness call physicians, but those who are suffering from a hopelessly incurable melancholy do not even admit the physicians if they come. May this now not be your plight, as would seem to be the case were you to shun these right counsels!  


\end{document}
